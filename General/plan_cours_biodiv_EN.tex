\documentclass[12]{article}
\usepackage[utf8]{inputenc}
\usepackage{authblk}
\usepackage{geometry}
\usepackage{amsmath}
\usepackage{hyperref}
\usepackage[french]{babel}
\usepackage{url}

\geometry{letterpaper,margin=2.5cm}

\hypersetup
{
    colorlinks = true, linkcolor = blue, citecolor = blue, urlcolor = blue,
}

%\def\labelitemi{$\bullet$}

\title{ECL 707 (MSc) - ECL 807 (PhD) \\ Summer School in Biodiveristy Modelling\\Orford Musique -- August 20--24, 2018}
\date {}
\author[1]{Dominique Gravel}
\author[1]{F. Guillaume Blanchet}
\author[2]{Matthew Talluto}
\affil[1]{Départment de biologie, Université de Sherbrooke}
\affil[2]{Leibniz-Institute of freshwater ecology and inland fisheries}

\begin{document}

	\maketitle

	%-----------------------------
	\section*{General Objective}

  Biodiversity modelling is now an integral part of the work of biologists in
  many subfields of biology from fundamental ecology to evolutionary biology and
  conservation biology. By the end of this course, the student will be able to
  use different biodiversity modelling techniques and understand the limits of
  different types of ecological data. The student will also be able to
  understand the basis of the different approaches used to model and predict
  different facets of the biodiversity (e.g. species distribution or trophic
  networks structure) and to use these approaches in his own projects. Finally,
  the student will also be able to apply the different techniques discussed
  during the course to decide and inform on the structure of biodiversity of a
  particular region.

	%-----------------------------
	\section*{Specific Objectives}

	By the end of this course, the student will :

	\begin{itemize}
	\renewcommand{\labelitemi}{$\bullet$}

  \item be able to use different biodiversity modelling techniques (statistics,
  differential equations, stochastic simulations)

  \item be able to use outputs of climate change models for biodiversity
  modelling

  \item be able to devise various biodiversity scenarios based on the analysis
  of empirical data

  \item have developed critical thinking about the methods used in different
  biodiversity scenarios

  \item be able to evaluate the source of uncertainty in predictions of
  biodiversity

	\end{itemize}

	%-----------------------------
	\section*{Prerequisites}

  To follow this course, intermediate knowledge of scientific programming is
  required.

  %-----------------------------

  \section*{Pedagogical Approach}

  For each topic covered, short lectures on the basic theory will be separated
  by practical sessions designed to apply what was learned in the lecture. The
  sections will end with an exercise highlighting the different aspects of the
  topic covered. Each topic will be complemented by discussions on the
  characteristics of the approaches discussed and their use in practice.
  Students will have a data analysis project to complete that will encompass all
  aspects of what has been seen during the summer school.

  All the course material will be made available on the following git
  repository:

  https://github.com/EcoNumUdS/ScenariosBiodiv

	%-----------------------------
	\section*{Topics}

  \subsection*{Day 1}
  \subsubsection*{Morning -- Introduction}
  \begin{itemize}
    \item Phenomenological vs mecanistic approches
    \item Use of biodiversity scenarios for decision makers
    \begin{itemize}
      \item Overview of tools used (DGVM, Forest Gap Models, SDM, SAR ...)
      \item Examples of biodiversity scenarios
    \end{itemize}
    \item Basis of modelling (ODE, stochastic, statistics)
    \item An example : The QUICC-FOR project
  \end{itemize}

  \subsubsection*{Afternoon -- Anthropogenic Change Scenarios}
  \begin{itemize}
    \item Climatic models
    \item Resource selection models
  \end{itemize}
  \subsubsection*{Exercice}
  Temperate forest migration simulations

  \subsubsection*{Seminar (optional)}
  Directed exercises on algorithmic
  \begin{itemize}
    \item ODE
    \item Stochastic simulations
  \end{itemize}

  \subsection*{Day 2 -- Spatial and Land Use Change Models}
  \subsubsection*{Morning -- Spatial dynamic}
  \begin{itemize}
    \item Stochastic simulations : Markov chains
    \item Metapopulation theory
    \item Cellular automata
  \end{itemize}

  \subsubsection*{Afternoon -- Connectivity}
  \begin{itemize}
    \item Landscape connectivity tools
    \item Using landscape occupations open data
  \end{itemize}
  \subsubsection*{Exercice}
  Spatial co-dynamic of forests and avian communities

  \subsubsection*{Seminar}
  Marie-Josée Fortin, University of Toronto

  \subsection*{Day 3 -- Biodiversity Distribution and Climate Change}
  \subsubsection*{Morning -- Species Distribution Models (SDMs)}
  \begin{itemize}
    \item Théorie
    \item MaxEnt
    \item BioMod and ensemble forecasting
  \end{itemize}
  \subsubsection*{Afternoon -- Using Open Data on Biodiveristy}
  \subsubsection*{Exercice}
  Mapping bird richness and uncertainty in the south of Québec

  \subsubsection*{Seminar}
  Anne Bruneau, Université de Montréal

  \subsection*{Day 4 -- Community Models}
  \subsubsection*{Morning -- Joint Species Distribution Models (JSDMs)}
  \begin{itemize}
    \item Theory
    \item Using the \texttt{HMSC} R package
  \end{itemize}
  \subsubsection*{Afternoon -- Networks Dynamics, Extinctions and Ecosystem Use}
  \begin{itemize}
    \item Robustness Analysis
    \item Ecosystem Dynamics
  \end{itemize}
  \subsubsection*{Exercice}
  \begin{itemize}
    \item Mapping bird richness and uncertainty in the south of Québec
    (revisited)
    \item Spatial dynamics of ecological networks
  \end{itemize}

  \subsubsection*{Seminar}
  Anouk Simard, Gouvernement du Québec

  \subsection*{Day 5 -- Decision Making}
  \subsubsection*{Morning -- Tools for Decision Makers}
  \begin{itemize}
    \item Evaluation of uncertainty
    \item Structural sensibility
    \item Optimization
  \end{itemize}
  \subsubsection*{Afternoon}
  \begin{itemize}
    \item Student's presentation
    \item Integrative exercises : optimizing a network of protected areas to
    maximize avian communities conservation
  \end{itemize}

	%-----------------------------
	\section*{Evaluation}

  The evaluation will be based on a class project to complete alone or in a
  team. A presentation of the project and of the model use will be made during
  the week. A final report, including the code used to perform all analyses,
  will need to be handed in at the latest by \textbf{\textcolor{red}{December
  31, 2018}}. Students will have the opportunity to use their own data or open
  access data to carry out their project.

\end{document}
